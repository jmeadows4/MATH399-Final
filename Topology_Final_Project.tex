\documentclass[12pt]{report}
\usepackage{latexsym}
\usepackage{amsmath}
\usepackage{amsfonts}
\usepackage{amssymb}
\usepackage{amsthm}

\setlength{\textwidth}{6.5in} 
\setlength{\textheight}{9in}
\setlength{\oddsidemargin}{-.1in} 
\setlength{\topmargin}{-1in}
\setlength{\parindent}{10ex}
\setlength{\parskip}{1em}
\usepackage{indentfirst}
\setcounter{MaxMatrixCols}{33}
\usepackage[utf8]{inputenc}
\usepackage{graphicx}
\graphicspath{ {./ } }

\begin{document}


\begin{titlepage}
   \begin{center}
       \vspace*{1cm}

       \textbf{Topology Final Project}

       \vspace{0.5cm}
        Topological Subtitle
            
       \vspace{1.5cm}

       \textbf{Josh Meadows, Alex Richards, Thomas Smale, Davie Coles}

       \vfill
            
       
            
       \vspace{0.8cm}
            
       Mathematics\\
       California State University, Chico\\
       United States\\
       28 April 2021
            
   \end{center}
\end{titlepage}

\section*{Table of Contents}

\clearpage
\section*{Introduction}

This paragraph begins the introduction to our final project for MATH 399 Topological Data Analysis. This is placeholder text used to show how paragraphs work in Latex. Now we will begin a new paragraph. \par
Here is a second paragraph. This is pretty easy. 

One of the most pressing matters on any college campus is climate change. Climate change is an issue that affects everyone in this world. As humans have evolved scientists believe that human actions are harming the planet. This is causing more extreme weather that is disrupting many natural habitats. As a result there may be enormous loss of both animal and human life. As students of a California State University, we have grown up in this beautiful state and come to appreciate the ground underneath us. We are concerned that human induced changes to the climate may result in our beautiful state turning to a dry desert. Working together and combining our backgrounds in math and computer science that we have learned at Chico State, we are going to apply groundbreaking methods to study our climate. One of the most beautiful regions in California is Lake Tahoe. It offers year round outdoor fun and our state depends on it for many resources. Since it is such a special place, people care about it and would hate to see it be destroyed. At 6,000 feet of elevation in the winter time it is cold and snowy, but warms up in the summertime with warm California blue skies. This diverse climate gives us many different aspects to analyze. The data comes from the National Oceanic and Atmospheric Administration, a federal government organization that documents climate all over the country. Government workers have been collecting daily data about the weather at its station in Tahoe City since 1903. This gives us over a 100 years of reliable data to analyze. These data points are available in a csv file that is about 43000 lines long. The data is multi dimensional as it has fields such as temperature, amount of precipitation, and 15 different weather types. We will combine all of these dimensions into a point cloud to study its shape. We will then analyze the data using topological data analysis to study its qualitative features. Topological data analysis is a great way to deal with multi dimensional data that traditional methods struggle to extract value from. We will also use traditional methods such as scatter plots or line plots to visualize the data. This is to help with our understanding of the data set and confirm our conclusions. However, we expect that using topological data analysis will provide us with information about the data not seen in the traditional graphs. Using data analysis we will be able to withdraw trends about the data like if the temperatures are increasing or the amount of yearly snow is decreasing. From this we will learn more about how the climate has been changing in Tahoe, which may be a reflection of a lot of California. 

---DAVID---
What are the problems to be solved? What are the mechanisms developed to solve the problems

 The amount of data being collected in the world is a massive value that is continuously increasing. Data can consist of our network traffic, social media accounts, and grocery receipts. Processing all of this data is an extreme challenge and those who can make sense of it are rewarded. The market size for data science is increasing as companies look to gain a competitive edge by utilizing data to make better decisions. Data analysis is no easy feat as the size of the data, noisiness, dimensions, and incompleteness cause challenges. There are many different ways to analyze data but in the last 15 years topological data analysis has been recognized as useful in dealing with high dimensional complex data. Topological data analysis is a combination of algebraic topology, computational geometry, computer science, and more. It measures the qualitative features of data by computing the persistent homology which utilizes algebraic topology. 


We are hoping to leverage a software package called dionysus. Dionysus is written by a student who studied under Gunnar Carlson, Edelsburg at Duke, and is currently at Lawrence Berkeley Laboratory. It is written in c++ with a python interface. C++ will give the library its speed, while python will provide a interface that is friendly to work with. 





\clearpage
\section*{Project}
\subsection*{Algorithm}
\includegraphics[scale=0.5]{balls_1_2.png}


\subsection*{Example 2}

\clearpage
\section*{Conclusion}

\clearpage
\section*{References}

\clearpage
\section*{Evaluation}

\clearpage
\end{document} 



